
\chapter{Linear-Combination Aggregator Obliviousness}\label{app:lcao_definition}
The following game between attacker and challenger defines the security notion of LCAO.
\begin{description}
    \item[Setup] The challenger chooses security parameter $\kappa$, runs the $\mathsf{Setup}(\kappa)$ algorithm and gives $\mathsf{pub}$, $m$ and $pk_0$ to the attacker
    \item[Queries] The attacker can now perform encryptions or submit queries that are answered by the challenger. The types of actions are:
    \begin{enumerate}
        \item \textit{Encryption:} The attacker chooses a value $x$ and computes an encryption of $x$ under the aggregator's public key $pk_0$, obtaining $\mathcal{E}_{pk_0}(x)$.
        \item \textit{Weight Queries:} The attacker chooses an instance $t$ and receives the weights for that instance encrypted with the aggregator's public key, $\mathcal{E}_{pk_0}(\omega^{(t)}_{j}),\,j\in\{1,\dots,m\}$.
        \item \textit{Combine Queries:} The attacker chooses a tuple $(i,t,a^{(t)}_{i,1},\dots,a^{(t)}_{i,m})$ such that for any two chosen combine query tuples $(i,t,a^{(t)}_{i,1},\dots,a^{(t)}_{i,m})$ and $(i',t',a^{\prime(t')}_{i',1},\dots,a^{\prime(t')}_{i',m})$, the following condition holds:
        \begin{equation*}
            i = i' \wedge t = t' \implies a^{(t)}_{i,j} = a^{\prime(t')}_{i',j},\,j\in\{1,\dots,m\}\,.
        \end{equation*}
        The attacker is then given back the encryption of the linear combination $\mathcal{E}_{pk_0,sk_i}(\sum^m_{j=1}a^{(t)}_{i,j}\omega^{(t)}_j)$ encrypted under both the aggregator public key $pk_0$ and the secret key $sk_i$.
        \item \textit{Compromise queries:} The attacker chooses $i$ and receives the secret key $sk_i$. The aggregator's secret key may also be compromised (when choosing $i=0$).
    \end{enumerate} 
    \item[Challenge] Next, the attacker chooses an instance $t^*$, and a subset of users $S \subseteq U$ where $U$ is the complete set of users for which no combine queries, for the instance $t^*$, and no compromise queries, are made for the duration of the game. The attacker then chooses two series of tuples
    \begin{equation*}
        \left\langle\left(i,t^*,a^{(t^*)(0)}_{i,1},\dots,a^{(t^*)(0)}_{i,m}\right)\,\middle|\,i \in S\right\rangle
    \end{equation*}
    and
    \begin{equation*}
        \left\langle\left(i,t^*,a^{(t^*)(1)}_{i,1},\dots,a^{(t^*)(1)}_{i,m}\right)\,\middle|\, i \in S\right\rangle\,,
    \end{equation*}
    and gives them to the challenger. In the case that $0 \in S$ (\textit{i.e.}, the aggregator is compromised) and $S = U$, it is additionally required that
    \begin{equation*}
        \sum_{i\in S}\sum^{m}_{j=1} a^{(t^*)(0)}_{i,j}\omega^{(t^*)}_j = \sum_{i \in S}\sum^{m}_{j=1} a^{(t^*)(1)}_{i,j}\omega^{(t^*)}_j\,,
    \end{equation*}
    for weights $\omega^{(t^*)}_j,\,j\in\{1,\dots,m\}$ returned by a \textit{Weight Query} with chosen instance $t^*$. The challenger then chooses a random bit $b \in \{1,0\}$ and returns encryptions 
    \begin{equation*}
        \left\langle\mathcal{E}_{pk_0,sk_i}\left(\sum^m_{j=1}a^{(t^*)(b)}_{i,j}\omega^{(t^*)}_j\right)\,\middle|\,i\in S\right\rangle\,.
    \end{equation*}
    \item[More Queries] The attacker can now perform more encryptions and submit queries, so long as the queries do not break the requirements in the Challenge stage. That is, $S \subseteq U$.
    \item[Guess] At the end, the attacker outputs a bit $b'$ and wins the game if and only if $b' = b$. The advantage of an attacker $\mathcal{A}$ is defined as
    \begin{equation*}
        \mathsf{Adv}^{LCAO}(\mathcal{A}) \coloneqq \left\lvert \Pr [b'=b] - \frac{1}{2}\right\rvert\,.
    \end{equation*} 
\end{description}

\begin{definition}
    An encryption scheme meets LCAO security if no probabilistic adversary, running in polynomial-time with respect to security parameter $\kappa$, has more than a negligible advantage in winning the above security game. That is, for all adversaries $\mathcal{A}$, there exists a negligible function $\eta$, such that
    \begin{equation*}
        \mathsf{Adv}^{LCAO}(\mathcal{A}) \leq \eta(\kappa)\,,
    \end{equation*}
    with probabilities taken over randomness introduced by $\mathcal{A}$, and in $\mathsf{Setup}$, $\mathsf{Enc}$ and $\mathsf{CombEnc}$.
\end{definition}

\chapter{Cryptographic Proof of LCAO Scheme Security}