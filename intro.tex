
\chapter{Introduction}
Sensor data processing, state estimation and data fusion have long been active areas of research and continue to find applications in modern systems \cite{andersonOptimalFiltering1979,simonOptimalStateEstimation2006}. As distributed networks have become more prevalent over the years, greater stress has been put on the need for broadly applicable algorithms that support varying types of measurements, estimate accuracies and communication availabilities \cite{mutambaraDecentralizedEstimationControl1998,ligginsDistributedDataFusion2012}, supporting use cases including localisation, weather forecasting, mapping, cooperative computing and cloud computing []. The use of Bayesian estimation methods such as the popular Kalman filter and its non-linear derivatives have become especially prevalent in application due to their recursive, often optimal, estimation properties and their suitability for modelling cross-correlations between local estimates \cite{chongFortyYearsDistributed2017,haugBayesianEstimationTracking2012}. The handling of these cross-correlations, especially when they are not known in advance, is a common difficulty in state estimation and is tied to the challenges within the field \cite{noackTreatmentDependentInformation2017}. Much of the work in this field, including some of the methods presented in this thesis, handle these challenges.

While the challenges in data fusion and state estimation that relate to correlation errors are a well-established field of research, widespread advancements in distributed computing and uses of public networks for sensor communication have led the additional requirements of data privacy and state secrecy to become particularly relevant in recent years \cite{brennerSecretProgramExecution2011,renSecurityChallengesPublic2012}.

Problems that require careful consideration of correlations errors while simultaneously guaranteeing a level of security for the participants involved are therefore

Typical cryptographic secrecy involves hiding all transferred data such that unauthorised parties gain no information from acquired encryptions. This can often be achieved irrespective of the estimation algorithms by using common symmetric and public-key encryption schemes such as the Advanced Encryption Standard (AES) [] and the Rivest-Shamir0Adleman cryptosystem (RSA) [], respectively.


---



more complicated examples have different requirements where these are not suitable

more complicated schemes

differential privacy as a statistical security

Due to the nature of cryptographic goals and proofs in distributed environments, security goals in estimation and fusion are often very context-specific and have led to numerous solutions for various scenarios and security desires

leads us into the state-of-the-art

% 
%  .d8888b.   .d88888b. 88888888888     d8888 
% d88P  Y88b d88P" "Y88b    888        d88888 
% Y88b.      888     888    888       d88P888 
%  "Y888b.   888     888    888      d88P 888 
%     "Y88b. 888     888    888     d88P  888 
%       "888 888     888    888    d88P   888 
% Y88b  d88P Y88b. .d88P    888   d8888888888 
%  "Y8888P"   "Y88888P"     888  d88P     888 
%                                             
%                                             
%                                             
% 
\section{State-of-the-Art and Research Questions}

As mentioned in the previous section, the nature of cryptographic notions in distributed environments typically requires communications between participating parties to be known exactly. This, in turn, has led to many, otherwise, general estimation algorithms, being restricted in some way to make communication and security easier to discuss.

note privacy-preserving expression in terms of data confidentiality

For example, [aristov] presents a distributed Kalman filter, namely an Information filter, where sensor measurements and measurement errors are kept private to the measuring sensors only, while the final estimation update (sum of this information) is leaked to an estimator. For this to be achieved, however, sensors must form a hierarchical communication structure and measurement models must be linear, restricting the otherwise more broadly applicable information filter and its non-linear variants.

[proloc]

Pisac and pwsah papers

aggregation papers

differentially private Kalman filtering

privacy-preserving optimisation with security based on statistical estimation

added noise estimation

--

privacy-preserving image-based localisation

eavesdropper paper with a secure return channel and a lossier channel for eavesdroppers

GPS

chaotic system paper

physical layer noise paper (similar to chaotic noise paper)

--


The two different approaches, restricting existing broad estimation methods in some ways to make cryptographic analysis plausible and ignoring formal security when assumptions and conclusions are intuitive demonstrate a gap in the existing literature and bring us to the target research topics this thesis aims to explore.

\begin{itemize}
    \item dot point topics
\end{itemize}

These broad topics aim to fulfil the goal of generalisable but cryptographically provable estimation and fusion methods in distributed environments and lead to the concrete problems tackled in this work

% 
%  .d8888b.   .d88888b.  888b    888 88888888888 
% d88P  Y88b d88P" "Y88b 8888b   888     888     
% 888    888 888     888 88888b  888     888     
% 888        888     888 888Y88b 888     888     
% 888        888     888 888 Y88b888     888     
% 888    888 888     888 888  Y88888     888     
% Y88b  d88P Y88b. .d88P 888   Y8888     888     
%  "Y8888P"   "Y88888P"  888    Y888     888     
%                                                
%                                                
%                                                
% 
\section{Contributions}

The contributions tackle the research topics in section .. by considering three concrete problems that coincide with the broader problems in the field



% 
%  .d8888b. 88888888888 8888888b.  888     888  .d8888b. 88888888888 
% d88P  Y88b    888     888   Y88b 888     888 d88P  Y88b    888     
% Y88b.         888     888    888 888     888 888    888    888     
%  "Y888b.      888     888   d88P 888     888 888           888     
%     "Y88b.    888     8888888P"  888     888 888           888     
%       "888    888     888 T88b   888     888 888    888    888     
% Y88b  d88P    888     888  T88b  Y88b. .d88P Y88b  d88P    888     
%  "Y8888P"     888     888   T88b  "Y88888P"   "Y8888P"     888     
%                                                                    
%                                                                    
%                                                                    
% 
\section{Thesis Structure}

Each chapter includes relevant related literature to the specific problem and a formal problem formalisation before presenting the novel solutions.