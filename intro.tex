
\chapter{Introduction}\label{ch:intro}
Sensor data processing, state estimation and data fusion have long been active areas of research and continue to find applications in modern systems \cite{andersonOptimalFiltering1979,simonOptimalStateEstimation2006}. As distributed networks have become more prevalent over the years, greater stress has been put on the need for broadly applicable algorithms that support varying types of measurements, estimate accuracies and communication availabilities \cite{mutambaraDecentralizedEstimationControl1998,ligginsDistributedDataFusion2012}, finding uses in localisation, weather forecasting, mapping, cooperative computing and more \cite{galanisApplicationsKalmanFilters2006,gillijnsWhatEnsembleKalman2006,geziciLocalizationUltraWidebandRadios2005,sieblerLocalizationMagneticField2020,kiaCooperativeLocalizationMobile2016,sridharCooperativePerceptionAutonomous2019,aulinasSLAMProblemSurvey2008}. In particular, handling cross-correlations between distributed data, especially when they are not known in advance, has been a well-studied difficulty in distributed estimation and is closely tied to the challenges in the field \cite{julierNondivergentEstimationAlgorithm1997,grimeDataFusionDecentralized1994,noackTreatmentDependentInformation2017,radtkeReconstructionCrossCorrelationsConstant2018}. The use of Bayesian estimation methods such as the popular Kalman filter and its non-linear derivatives have become especially prevalent in these applications due to their recursive, often optimal, properties and their suitability for modelling these cross-correlations \cite{chongFortyYearsDistributed2017,haugBayesianEstimationTracking2012,willnerKalmanFilterAlgorithms1976,pfaffInformationFormDistributed2017}. In recent years, widespread advancements in distributed algorithms and the ubiquity of public networks such as the Internet, wireless communication channels and the Internet-of-things (IoT) paradigm, have brought privacy challenges into focus as well \cite{brennerSecretProgramExecution2011,renSecurityChallengesPublic2012}. In particular, the data confidentiality component of the cryptographic Confidentiality-Integrity-Availability (CIA) triad \cite{keyserSecurityPolicy2005} has become an important goal in security-aware distributed data processing tasks. That is for concrete data, private to participants, to remain confidential or leakage to be formally quantifiable. In general, the broader topic of data \textit{privacy}, concerned with the identification of individuals by any means including the observation of this data, is used synonymously in literature \cite{farokhiPrivacyDynamicalSystems2020,specialePrivacyPreservingImageBased2019,erkinPrivacyPreservingFaceRecognition2009,hePreservingDataPrivacyAdded2018,liPrivacyPreservingDistributedOptimization2020} but will not be considered in its entirety in this thesis.

Traditional data confidentiality involves keeping transmitted information private from unauthorised parties in untrusted networks and can often be achieved irrespective of the data processing algorithms used. Typically, these scenarios can be achieved by using common symmetric and asymmetric encryption schemes such as the Advanced Encryption Standard (AES) \cite{gueronIntelAdvancedEncryption2010} or the Rivest-Shamir-Adleman (RSA) cryptosystem \cite{rivestMethodObtainingDigital1978}, respectively. These scenarios, however, imply trust between encrypting and decrypting parties, which cannot always be assumed in distributed environments. Situations where partial results are considered private, or only partial leakage of data is desired for computing results, do not assume this trust and have led to the development of several encryption schemes that provide encrypted operations and explicit formal leakages \cite{paillierPublicKeyCryptosystemsBased1999,shiPrivacyPreservingAggregationTimeSeries2011,chotardDecentralizedMultiClientFunctional2018,andresGeoIndistinguishabilityDifferentialPrivacy2013}. A very applicable group of these schemes in estimation, homomorphic encryption (HE), allow operations to be performed on encrypted data without decryption. These schemes can be loosely grouped into two categories: fully homomorphic encryption (FHE), allowing arbitrary operations on encryptions; and partially homomorphic encryption (PHE), allowing only a subset, typically one, operations. Although FHE suits a wider variety of estimation problems, essentially allowing arbitrary computations while preserving data confidentiality, its current implementations are still too computationally expensive for large-scale or real-time processing \cite{acarSurveyHomomorphicEncryption2018,gentryFullyHomomorphicEncryption2009,stehleFasterFullyHomomorphic2010}. For this reason, PHE has been the more popular choice in providing data confidentiality during a variety of estimation tasks \cite{lagendijkEncryptedSignalProcessing2012,ryanPretVoterPaillier2008,kerschbaumOutsourcedPrivateSet2012,catalanoUsingLinearlyHomomorphicEncryption2015,abdallaSingleInputMulticlientInnerProduct2019} and is predominantly relied on throughout this thesis. While these schemes provide a powerful tool for designing data-processing algorithms, the nature of cryptographic analysis in distributed environments depends heavily on communication protocols between participants, limiting the ease of their combination with general estimation and data fusion solutions such as the Bayesian methods mentioned previously. In turn, this has led to various context-specific estimation solutions with differing degrees of cryptographic guarantees, often restricting general solutions to provide meaningful cryptographic guarantees or foregoing provable security for more general algorithms. This leads us to the goals of this thesis and the current state-of-the-art in security-oriented estimation and data fusion.

% 
%  .d8888b.   .d88888b. 88888888888     d8888 
% d88P  Y88b d88P" "Y88b    888        d88888 
% Y88b.      888     888    888       d88P888 
%  "Y888b.   888     888    888      d88P 888 
%     "Y88b. 888     888    888     d88P  888 
%       "888 888     888    888    d88P   888 
% Y88b  d88P Y88b. .d88P    888   d8888888888 
%  "Y8888P"   "Y88888P"     888  d88P     888 
%                                             
%                                             
%                                             
% 

\section{Research Questions and the State-of-the-Art}\label{sec:intro:sota}
The restrictions on the generality of solutions and the frequent foregoing of cryptographic guarantees when providing security in estimation tasks form the literature gap that this thesis is centred around. The overarching topics we are interested in are as follows. We wish to find distributed estimation and data fusion solutions based on the Kalman filter for non-linear models with provable security. Here, non-linear models capture the broadest, and therefore most generally applicable, solutions in estimation. Secondly, we are interested in formalising novel cryptographic definitions that capture suitable communication protocols and leakages for any of these solutions should they not exist. Lastly, we would like to define a general cryptographic notion that captures adversary estimation performance and can be applied to existing security-aware estimation schemes with no cryptographically provable guarantees. From these topics, we concentrate on three specific problems that will form the main chapters of this thesis and discuss the state-of-the-art in the context of each.

% 
% ######## ##     ##  ######  ####  #######  ##    ## 
% ##       ##     ## ##    ##  ##  ##     ## ###   ## 
% ##       ##     ## ##        ##  ##     ## ####  ## 
% ######   ##     ##  ######   ##  ##     ## ## ## ## 
% ##       ##     ##       ##  ##  ##     ## ##  #### 
% ##       ##     ## ##    ##  ##  ##     ## ##   ### 
% ##        #######   ######  ####  #######  ##    ## 
% 

\subsection{Estimate Fusion on an Untrusted Cloud}\label{subsec:intro:conf_est_fusion}
The first problem we consider is confidential estimate fusion on a centralised untrusted cloud. This is a popular scenario in distributed sensor fusion, where a cloud or fusing party obtains estimates from within the network and fuses them centrally, providing a resulting fused state estimate for further processing \cite{ligginsDistributedFusionArchitectures1997,bar-shalomEffectCommonProcess1986,heRangeFreeLocalizationSchemes2003,congOrderInsensitiveSequential2016,yonggangxuEstimationUncontrolledControlled2005,rosenthalSchedulingMeasurementTransmission2018}. Some use cases for the scenario include factory sensor data fusion, object tracking, centralised weather forecasting, etc. Intuitively, untrusted cloud processing brings security concerns to mind, such as the confidentiality of individual estimate data and the privacy of participants producing it \cite{renSecurityChallengesPublic2012,leiOutsourcingLargeMatrix2013,nandakumarProtectingGridTopology2019,zhangSecureDotProduct2017,schulzedarupEncryptedControlNetworked2021}. Work on security-aware cloud processing and data fusion exists in a variety of scenarios. FHE and PHE are particularly suited to the problem, allowing computations to be finalised on confidential data before being queried by trusted parties for final results \cite{lopez-altOntheflyMultipartyComputation2012,poteyHomomorphicEncryptionSecurity2016,zhaoCloudComputingSecurity2014}. This includes control aggregation \cite{alexandruEncryptedCooperativeControl2019}, private matrix multiplication \cite{kogisoCyberSecurityEnhancementNetworked2015} and private set intersection \cite{kerschbaumOutsourcedPrivateSet2012}. Another relevant topic is differential privacy \cite{dworkDifferentialPrivacySurvey2008,wilsonDifferentiallyPrivateSQL2019}. Here, a formal cryptographic notion guarantees that individual inputs to data fusion cannot be exactly estimated by guaranteeing that results are indistinguishable when differing by only a single input. The downside to this cryptographically meaningful and often applicable solution is the noisiness of fusion results, rendering it unsuitable for scenarios where result accuracy cannot be compromised. We are interested in accurate general solutions to data fusion in a Bayesian setting and our solution to this problem aims to fuse arbitrary (non-linear and dependent) state estimates while a cryptographically meaningful assessment of confidentiality can be provided. Some applicable methods for this exist, albeit restricting the estimation or security requirements. In \cite{daruEncryptedCloudbasedControl2019}, control inputs can be computed in an encrypted control loop, with methods applicable to estimation, but rely on the presence of two clouds that cannot maliciously collude. \cite{aristovEncryptedMultisensorInformation2018} presents a method for homomorphic fusion but requires that partial fusions are collected in a hierarchical network and for fused measurements to be linear and independent. While in \cite{alanwarPrOLocResilientLocalization2017}, the homomorphic fusion of data is used to perform range-measurement localisation on confidential measurements but does not lend itself to a Bayesian setting where measurement noise properties are considered. The formalised estimation problem and cryptographic goals as well as our novel solutions to this problem are presented in chapter \ref{ch:cloud_fusion}.

% 
% ##    ##  #######  ##    ## ##       #### ##    ## 
% ###   ## ##     ## ###   ## ##        ##  ###   ## 
% ####  ## ##     ## ####  ## ##        ##  ####  ## 
% ## ## ## ##     ## ## ## ## ##        ##  ## ## ## 
% ##  #### ##     ## ##  #### ##        ##  ##  #### 
% ##   ### ##     ## ##   ### ##        ##  ##   ### 
% ##    ##  #######  ##    ## ######## #### ##    ## 
% 

\subsection{Non-Linear Measurement Fusion with Untrusted Participants}\label{subsec:intro:conf_nonlin_measurements}
The next problem we look at is confidential measurement fusion when participants in the network are untrusted. The scenario is in principle similar to the fusion problem in section \ref{subsec:intro:conf_est_fusion} but is distinguished by the party using the fusion result and the properties of the measurements. Unlike using a cloud, final computed fusion results are often needed by the party computing the fusion itself, such as self-localisation and decentralised estimation \cite{sridharCooperativePerceptionAutonomous2019,grimeDataFusionDecentralized1994,pintoSelflocalisationIndoorMobile2013}, rendering HE methods less practical as no trusted querying party is present. We also distinguish measurements from estimates in that we assume they are independent, allowing for the use of accurate Bayesian estimation methods such as the Kalman filter which make this assumption \cite{haugBayesianEstimationTracking2012}. Methods \cite{aristovEncryptedMultisensorInformation2018} and \cite{alanwarPrOLocResilientLocalization2017} again tackle a similar problem, but remain limited in requiring a hierarchical communication network and not considering measurement noise properties, respectively. Similarly to the fusion problem above, differential privacy \cite{dworkDifferentialPrivacySurvey2008} is again a related and applicable field, including existing applications to the Kalman filter \cite{nyDifferentiallyPrivateKalman2012}, but results remain noisy and are considered undesirable when accuracy is important. To provide an accurate non-linear distributed estimation filter with meaningful cryptographic assessment, other cryptographic notions need to be considered. Cryptographic constructs that support homomorphic computations as well as the leakage of final results, as would be required in the case where a third-party cloud is not used, exist. Private Weighted Sum Aggregation with centralised or hidden weights (pWSAc and pWSAh, respectively) are introduced in \cite{schulzedarupEncryptedCooperativeControl2019} and \cite{alexandruPrivateWeightedSum2020} as a means for computing control inputs in a distributed network without leaking individual contributions. Here, formal definitions with different communication assumptions to those suitable in a non-linear estimation problem are given. Similarly, more definitions of aggregation schemes have been introduced \cite{shiPrivacyPreservingAggregationTimeSeries2011,joyeScalableSchemePrivacyPreserving2013,benhamoudaNewFrameworkPrivacyPreserving2016,darcoProbabilisticSecretSharing2018,beckerRevisitingPrivateStream2018,chanPrivacyPreservingStreamAggregation2012} with a variety of specified communication protocols. Again, a formalised estimation and cryptographic goal for this problem as well as the presented solutions will be shown in chapter \ref{ch:nonlin_fusion}.

% 
% ########  ######  ########    ########  #### ######## ######## 
% ##       ##    ##    ##       ##     ##  ##  ##       ##       
% ##       ##          ##       ##     ##  ##  ##       ##       
% ######    ######     ##       ##     ##  ##  ######   ######   
% ##             ##    ##       ##     ##  ##  ##       ##       
% ##       ##    ##    ##       ##     ##  ##  ##       ##       
% ########  ######     ##       ########  #### ##       ##       
% 

\subsection{Provable Estimation Difference}\label{subsec:intro:provable_est_perf}
The last problem that we consider is the creation of a cryptographic notion for proving the estimation difference between possible estimators. Use cases are varied and include scenarios where a provable difference is desired, including the difference between trusted and untrusted estimators as well as the difference between parties of different priorities or with differing access to data. In many works, degraded estimation performance at untrusted parties is used as a form of security in a system \cite{specialePrivacyPreservingImageBased2019,liPrivacyPreservingDistributedOptimization2020,leongTransmissionSchedulingRemote2019,leongInformationBoundsState2019,grovesPrinciplesGNSSInertial2015}. These include additional communication channels \cite{leongInformationBoundsState2019,grovesPrinciplesGNSSInertial2015}, adding noise that is removable by trusted estimators \cite{murguiaInformationTheoreticPrivacyChaos2020,leongUseArtificialNoise2018} as well as noise that doesn't affect the final goal of estimation \cite{liPrivacyPreservingDistributedOptimization2020}. It is not uncommon for the performance difference between trusted and untrusted estimators to be analysed in these works \cite{hePreservingDataPrivacyAdded2018,murguiaInformationTheoreticPrivacyChaos2020,sinopoliKalmanFilteringIntermittent2004,mishraSecureStateEstimation2015} but is typically done so from an information-theoretic point of view, neglecting the generation of noise and assuming real-number representations are exact in practice. Differential privacy \cite{dworkDifferentialPrivacySurvey2008} is again relevant in this problem as it captures the inability to estimate a missing piece of data from a cryptographic perspective. However, it targets the confidentiality of contributions to statistical data and does not capture the imperfect ability of a trusted estimator to estimate a Bayesian system. We aim to capture a cryptographically provable difference in performance between trusted and untrusted estimators in a Bayesian setting while taking into account the computational capabilities of attackers. In addition, we aim to present a scheme to which this notion can be applied. This problem, its cryptographic goals and presented solutions are formalised in chapter \ref{ch:priv_estimation}.

% 
%  .d8888b.   .d88888b.  888b    888 88888888888 
% d88P  Y88b d88P" "Y88b 8888b   888     888     
% 888    888 888     888 88888b  888     888     
% 888        888     888 888Y88b 888     888     
% 888        888     888 888 Y88b888     888     
% 888    888 888     888 888  Y88888     888     
% Y88b  d88P Y88b. .d88P 888   Y8888     888     
%  "Y8888P"   "Y88888P"  888    Y888     888     
%                                                
%                                                
%                                                
% 

\section{Structure and Contributions}\label{sec:intro:contributions}
After this introductory chapter, preliminaries for all content in the thesis are introduced in chapter \ref{ch:prelims}. The following chapters are independent, each tackling one of the problems aimed to be solved and presenting appropriate contributions. Each of these chapters includes an individual problem formulation and conclusion. 

Chapter \ref{ch:cloud_fusion} focuses on the problem of estimate fusion on an untrusted cloud, initially introduced in section \ref{subsec:intro:conf_est_fusion}, and makes the following contributions:
\begin{itemize}
    \item A novel method for the fusion of arbitrary stochastic estimates at a cloud while keeping individual estimates confidential and leaking only fusion weights at the cloud.
    \item A novel method for the fusion of arbitrary stochastic estimates at a cloud while keeping both fusion weights and individual estimates confidential at the cloud.
    \item Cryptographic analysis and simulations of both presented methods.
\end{itemize}

Chapter \ref{ch:nonlin_fusion} looks at non-linear measurement fusion with untrusted participants, introduced in section \ref{subsec:intro:conf_nonlin_measurements}, and presents the following:
\begin{itemize}
    \item A novel cryptographic notion capturing the confidential linear combination of weights in a distributed network.
    \item A novel encryption scheme meeting the defined notion.
    \item A novel modification to a distributed estimation filter that allows range-only localisation using the defined encryption scheme such that estimates, sensor measurements and sensor properties remain confidential.
    \item An analysis of the method's extension to more general non-linear environments.
    \item A cryptographic proof for the defined encryption scheme meeting the defined cryptographic notion.
    \item A cryptographic analysis and simulations of the presented estimation method.
\end{itemize}

Chapter \ref{ch:priv_estimation} focuses on the problem of provable estimation differences, introduced in section \ref{subsec:intro:provable_est_perf}, and presents:
\begin{itemize}
    \item A novel cryptographic notion capturing the difference in performance between estimators in a Bayesian estimation scenario.
    \item A novel single-sensor estimation scheme that relies on a secret key to distinguish between types of estimators with performance differences provable with the defined notion.
    \item An extension to a multiple-sensor estimation scheme with similar properties that takes the fusion of measurements into account.
    \item A cryptographic analysis and simulations of the presented methods.
\end{itemize}

Finally, chapter \ref{ch:conclusion} concludes the thesis in its entirety.