
\chapter{Introduction}
Sensor data processing, state estimation and data fusion have long been active areas of research and continue to find applications in modern systems \cite{andersonOptimalFiltering1979,simonOptimalStateEstimation2006}. As distributed networks have become more prevalent over the years, greater stress has been put on the need for broadly applicable algorithms that support varying types of measurements, estimate accuracies and communication availabilities \cite{mutambaraDecentralizedEstimationControl1998,ligginsDistributedDataFusion2012}, finding uses in localisation, weather forecasting, mapping, cooperative computing and cloud computing []. The use of Bayesian estimation methods such as the popular Kalman filter and its non-linear derivatives have become especially prevalent in application due to their recursive, often optimal, estimation properties and their suitability for modelling cross-correlations between local estimates \cite{chongFortyYearsDistributed2017,haugBayesianEstimationTracking2012}. The handling of these cross-correlations, especially when they are not known in advance, is a common difficulty in state estimation and is tied to the challenges within the field \cite{noackTreatmentDependentInformation2017}. The methods presented in this thesis and much of the related work in data fusion tasks similarly require the consideration of these challenges.

While the challenges relating to correlation errors are a well-established field of research, widespread advancements in distributed computing and uses of public networks for sensor communication have put a focus on the additional requirements of data privacy and state secrecy in recent years as well \cite{brennerSecretProgramExecution2011,renSecurityChallengesPublic2012}. Problems that require both the handling of correlation errors and guaranteeing a level of security for the participants involved are therefore a relevant topic in state estimation today and at the core of the work presented in this thesis. Achieving cryptographic secrecy typically involves hiding transferred information from unauthorised parties and can often be achieved irrespective of the estimation algorithms used by using common symmetric and public-key encryption schemes such as the Advanced Encryption Standard (AES) \cite{gueronIntelAdvancedEncryption2010} and the Rivest-Shamir-Adleman cryptosystem (RSA) \cite{rivestMethodObtainingDigital1978}, respectively. These scenarios, however, imply a trust between encrypting and decrypting parties, which cannot always be assumed in distributed environments. In addition, partial computations on encrypted data or the intended leakage of some results are sometimes required for computing final results. This has led to several operation-providing and leakage-supporting encryption schemes \cite{paillierPublicKeyCryptosystemsBased1999,shiPrivacyPreservingAggregationTimeSeries2011,joyeScalableSchemePrivacyPreserving2013,chotardDecentralizedMultiClientFunctional2018,andresGeoIndistinguishabilityDifferentialPrivacy2013} suitable for these distributed environments or when fine control over leakage is required. While these cryptographic schemes and notations are applicable in estimation and fusion tasks, the nature of cryptographic analysis in distributed environments, heavily dependent on communication protocols, has meant that developed solutions are often very context-specific, leading to numerous solutions for various estimation scenarios, use-cases and security requirements. This leads us to the current state-of-the-art literature on security-oriented state estimation and data fusion, observable gaps in this literature and the research questions we aim to answer.


% 
%  .d8888b.   .d88888b. 88888888888     d8888 
% d88P  Y88b d88P" "Y88b    888        d88888 
% Y88b.      888     888    888       d88P888 
%  "Y888b.   888     888    888      d88P 888 
%     "Y88b. 888     888    888     d88P  888 
%       "888 888     888    888    d88P   888 
% Y88b  d88P Y88b. .d88P    888   d8888888888 
%  "Y8888P"   "Y88888P"     888  d88P     888 
%                                             
%                                             
%                                             
% 
\section{State-of-the-Art and Research Questions}
% Makes sense to move this to the introduction above, no?
To summarise relevant work in security-oriented state estimation and data fusion we first discuss the security and types of data explored in this thesis. Regarding algorithms, we primarily consider stochastic models and outputs. As discussed in section [], Bayesian estimation methods such as the Kalman filter are prevalent due to their applicability and suitability for modelling phenomena accurately. In terms of security, we restrict ourselves to the data confidentiality component of the Confidentiality-Integrity-Availability (CIA) triad []. That is, the primary concern of security in this thesis is that concrete data deemed to be private to some participants remain so and its leakage is formally quantifiable. Data privacy, a related concept, is concerned with stopping the identification of individuals from available information. Although data privacy encompasses data confidentiality, it is a broader topic including communication traffic analyses and external cross-referencing to identify individuals and is not considered in its entirety in this thesis. That said, the terms \textit{data privacy} and \textit{privacy-preserving} are often used in the state-of-the-art to refer to data confidentiality alone and the ability to identify individuals from concrete data available []. The terms will be similarly used throughout this thesis. 

Since knowing exact communications between participating parties is required for meaningful cryptographic analysis of transferred data, many existing general estimation algorithms have been restricted in some way to make communication and security easier to discuss. For example, [aristov] presents a distributed Kalman filter, namely an Information filter, where sensor measurements and measurement errors are known only to the measuring sensors while final estimates are leaked to an estimator. The restriction for this to be achieved requires sensors to form a hierarchical communication structure and measurement models to be linear, limiting the otherwise broadly applicable non-linear models or arbitrary communication structures. Another work, [proloc], presents localisation using range-only measurements where measurements and sensor locations, both required for localisation, are kept private to sensors and a centralised estimator while final estimates are available to an external trusted party. Here, no communication structure is enforced but produced estimates provide no error statistics and do not consider a dynamic process model.

In [pwsac, pwsah], 



aggregation papers

differential privacy and differentially private Kalman filtering

privacy-preserving optimisation with security based on statistical estimation

added noise estimation

--

privacy-preserving image-based localisation

eavesdropper paper with a secure return channel and a lossier channel for eavesdroppers

GPS

chaotic system paper

physical layer noise paper (similar to chaotic noise paper)

--


The two different approaches, restricting existing broad estimation methods in some ways to make cryptographic analysis plausible and ignoring formal security when assumptions and conclusions are intuitive demonstrate a gap in the existing literature and bring us to the target research topics this thesis aims to explore.

\begin{itemize}
    \item dot point topics
\end{itemize}

These broad topics aim to fulfil the goal of generalisable but cryptographically provable estimation and fusion methods in distributed environments and lead to the concrete problems tackled in this work

% 
%  .d8888b.   .d88888b.  888b    888 88888888888 
% d88P  Y88b d88P" "Y88b 8888b   888     888     
% 888    888 888     888 88888b  888     888     
% 888        888     888 888Y88b 888     888     
% 888        888     888 888 Y88b888     888     
% 888    888 888     888 888  Y88888     888     
% Y88b  d88P Y88b. .d88P 888   Y8888     888     
%  "Y8888P"   "Y88888P"  888    Y888     888     
%                                                
%                                                
%                                                
% 
\section{Contributions}

The contributions tackle the research topics in section .. by considering three concrete problems that coincide with the broader problems in the field



% 
%  .d8888b. 88888888888 8888888b.  888     888  .d8888b. 88888888888 
% d88P  Y88b    888     888   Y88b 888     888 d88P  Y88b    888     
% Y88b.         888     888    888 888     888 888    888    888     
%  "Y888b.      888     888   d88P 888     888 888           888     
%     "Y88b.    888     8888888P"  888     888 888           888     
%       "888    888     888 T88b   888     888 888    888    888     
% Y88b  d88P    888     888  T88b  Y88b. .d88P Y88b  d88P    888     
%  "Y8888P"     888     888   T88b  "Y88888P"   "Y8888P"     888     
%                                                                    
%                                                                    
%                                                                    
% 
\section{Thesis Structure}

Each chapter includes relevant related literature to the specific problem and a formal problem formalisation before presenting the novel solutions.