
\chapter{Conclusion}\label{ch:conclusion}
With the considered problems described and solutions presented, we can conclude this thesis with a brief discussion of the contributions to security-aware distributed estimation and future directions of the field. Chapters \ref{ch:cloud_fusion} and \ref{ch:nonlin_fusion} aimed to present general solutions to common estimation problems while taking into consideration the confidentiality of transmitted data. Difficulties achieving this were first made apparent in chapter \ref{ch:cloud_fusion}, where fusion was computed using PHE by either leaking fusion weights or by requiring extra computation upon decryption, noting that when no leakage was present estimates could not be prioritised as may sometimes be beneficial. Similarly, solutions presented in chapter \ref{ch:nonlin_fusion} tackled a specific non-linear problem rather than estimation with arbitrary non-linear measurements due to the requirement of a fixed communication protocol when proving cryptographic guarantees in a distributed environment. The provided extension of the solution to any models that can be written in the required protocol gave a more general solution, albeit not arbitrary. Although with limitations, it is clear that the novel solutions provide methods for data-confidential distributed estimation where previous general solutions did not exist. Chapter \ref{ch:priv_estimation} tackled a different problem, defining a general cryptographic notion for the difference between estimators that differ in the measurements they observe while taking into account the computational capabilities of attackers. As well as the presented new schemes that can use the notion to prove estimation differences, the generality of the definition allows its application to existing schemes where optimal or near-optimal estimators are known, creating formal security guarantees where ones did not previously exist. Following the methodology of splitting problems into estimation and cryptographic components solved separately, all three chapters present successful results, aiming to form a basis for future work, with the hope of even more general security-aware solutions. As hardware advancements are made and up-and-coming cryptographic solutions such as FHE are further developed, we look forward to interesting developments in the field that may build upon the methods presented here.